\documentclass{beamer}
\usepackage{fancyvrb}
\usepackage{mathtools}
\usepackage{amsmath}
\usepackage{xcolor}
\usepackage{tikz}
\usepackage{graphicx}

\usetikzlibrary{calc}

\usecolortheme{rose}
\usefonttheme{structurebold}
\setbeamercovered{again covered=\opaqueness<1->{50}}
\setbeamertemplate{navigation symbols}{}

\newcommand{\highlight}[1]{%
  \colorbox{red!50}{$\displaystyle#1$}}

\newcommand{\beginner}{
  \begin{tikzpicture}[remember picture, overlay]
    \node at ($(current page.north east)+(-1cm,-1cm)$) {\includegraphics[width=1cm]{thing.png}};
  \end{tikzpicture}
}

\title[pbkdf2]{PBKDF2: performance matters}
\author{Joseph Birr-Pixton\\
@jpixton\\
http://jbp.io/}
\date{}

\begin{document}

\frame{\titlepage}


\frame
{
  \frametitle{PBKDF2: quick intro} \beginner

  \begin{block}{Purpose}<1>
    Slowly convert a password + salt into a symmetric key of some length
  \end{block}

  \begin{block}{Origin}<2>
    RSA labs, 1999. Described in PKCS\#5 and then RFC2898
  \end{block}
}

\frame
{
  \frametitle{PBKDF2: quick intro} \beginner
  \begin{block}{Usage}<1>
    \begin{itemize}
      \item Password verification (web sites, network services, etc.)
      \item Key derivation (disk encryption, key management, etc.)
    \end{itemize}
  \end{block}

  \begin{alertblock}{Performance}<2>
    Performance profile is \emph{important} for defenders.  Aim: to
    maximise attacker work for defender computation budget.
  \end{alertblock}

  \begin{exampleblock}{Simplification}<3>
    PBKDF2 can produce arbitrary length output.

    We're going to ignore this capability from here on in: only considering
    the first block of output.
  \end{exampleblock}
}

\frame
{
  \frametitle{PBKDF2: how it was described} \beginner

  \begin{align*}
    \text{PBKDF2}_\text{PRF}(\text{pw}, \text{salt}, \text{i}) &\coloneqq U_1 \oplus U_2 \oplus \cdots \oplus U_\text{i} \\
    \onslide<2->{
      \text{where} \\
      U_1 &\coloneqq \text{PRF}(\text{pw}, \text{salt}\ \Vert\ 0_{32}) \\
      U_n &\coloneqq \text{PRF}(\text{pw}, U_{n-1}) \\
    }
    \onslide<3->{
      \text{and typically} \\
      \text{PRF}(\text{pw}, \text{x}) &= \text{HMAC-H}(\text{pw}, \text{x}) \\
      H &= \text{SHA-1, SHA-256, SHA-512, or ...}
    }
  \end{align*}
}

\frame
{
  \frametitle{Zoom, enhance}

  The function $\text{PBKDF2}_{\text{HMAC-SHA-256}}$ is slow because it
  executes the SHA-256 compression function many times.

  \begin{block}{How many times?}<2->
    Assumption: password and salt much shorter than SHA-256's 64-byte block size.
  \begin{align*}
    \text{HMAC-H}(k, m) &\coloneqq
      \text{H}(\alert<5>{k \oplus \text{opad}}\ \Vert\ 
      \alert<6>{\text{H}(\alert<3>{k \oplus \text{ipad}}\ \Vert\ \alert<4>{m})}) \\
    \onslide<3>{\text{block 1} &: k \oplus \text{ipad}} \\
    \onslide<4>{\text{block 2} &: m} \\
    \onslide<5>{\text{block 3} &: k \oplus \text{opad}} \\
    \onslide<6>{\text{block 4} &: \text{block 2 output}}
  \end{align*}

  \uncover<7->{Therefore, we need to compute $4\text{i}$ SHA-256 blocks.}
  \end{block}
}

\frame
{
  \frametitle{Nope!}

  This is actually wrong.  Neither PKCS\#5 nor RFC2898 mention this, or
  describe the expected performance.
  \begin{align*}
    \onslide<2->{
    & U_1 \oplus U_2 \oplus \cdots \oplus U_\text{i} \\
    }
    \onslide<3->{
      \text{with} \\
      U_1 &\coloneqq \text{HMAC-H}(\text{pw}, \text{salt}\ \Vert\ 0_{32}) \\
      U_n &\coloneqq \text{HMAC-H}(\text{pw}, U_{n-1}) \\
    }
    \onslide<4->{
      \text{(or equivalently)} \\
      U_1 &\coloneqq \text{H}(\alert<5>{\text{pw} \oplus \text{opad}}\ \Vert\ \text{H}(\alert<5>{pw \oplus \text{ipad}}\ \Vert\ \text{salt}\ \Vert\ 0_{32})) \\
      U_n &\coloneqq \text{H}(\alert<5>{\text{pw} \oplus \text{opad}}\ \Vert\ \text{H}(\alert<5>{pw \oplus \text{ipad}}\ \Vert\ U_{n-1}))
    }
  \end{align*}

  \uncover<5->{We can compute these blocks once}

  \begin{block}{How many times?}<6->
    Actually, we only need compute $2 + 2i$ SHA-256 blocks.
  \end{block}
}

\frame
{
  \frametitle{Our survey says...}

  \begin{columns}[T]
    \column{.48\textwidth}
    \begin{exampleblock}{Good: compute $2 + 2i$ blocks}
      \begin{itemize}
        \item SJCL
        \item<2-> OpenSSL (after Nov 2013)
        \item<2-> Python core ($\ge$3.4)
        \item<2-> Django (CVE-2013-1443, sc00bz)
        \item<2-> BouncyCastle ($\ge$1.49)
      \end{itemize}
    \end{exampleblock}
   
    \uncover<3->{
    \begin{alertblock}{Bad: compute $4i$ blocks}
      \begin{itemize}
        \item FreeBSD
        \item GRUB
        \item Android (BouncyCastle)
      \end{itemize}
    \end{alertblock}
    }

    \column{.48\textwidth}
    \uncover<4->{
    \begin{alertblock}{Bad: compute $4i$ blocks}
      \begin{itemize}
        \item Python (pypi pbkdf2)
        \item Ruby (pbkdf2 gem)
        \item Go (go.crypto)
        \item OpenBSD
        \item PolarSSL
        \item CyaSSL
        \item Java (OpenJDK)
        \item Common Lisp (ironclad)
        \item Perl (Crypt::PBKDF2)
        \item PHP
        \item C\#
      \end{itemize}
    \end{alertblock}
    }

  \end{columns}
}

\frame
{
  \frametitle{Selected performance measurements}

  \begin{itemize}
    \item<1> Question: how much practical difference does this make?
    \item<2> Let's measure PBKDF2-HMAC-SHA1 for large iteration count ($2^{22}$)
  \end{itemize}
}

\frame
{
  \frametitle{Selected performance measurements}

  \input measurements.tex

  \begin{figure}
  \centering

  \begin{tikzpicture}[
    y = .5cm,
    x = .7cm,
    font = \small\sffamily,
    fast/.style = {circle, draw=green!50, fill=green!20, thick, inner sep=0.8mm, outer sep=1mm},
    maybefast/.style = {circle, draw=green!25, fill=green!10, thick, inner sep=0.8mm, outer sep=1mm},
    slow/.style = {circle, draw=red!50, fill=red!20, thick, inner sep=0.8mm, outer sep=1mm},
    thing/.style = {font=\footnotesize},
    maybe/.style = {draw=black!50},
  ]
    \draw (0,0) -- coordinate (y axis mid) (0,10);
    \draw (0,0) -- (14,0);
    \foreach \y in {0,1,...,10} {
      \draw (0pt,\y) -- (-3pt,\y) node[anchor=east] {\y};
      \draw (0,\y) -- (14,\y) [gray!30];
    }
    \foreach \y in {0,1,...,9} {
      \draw (0pt,\y+0.5) -- (-1pt,\y+0.5);
    }
    \node[rotate=90, above=0.8cm] at (y axis mid) {Seconds};

    \node[fast,label=right:\openssltime] at (1,\openssltime) {};
    \node[thing] at (1,-1) {OpenSSL};

    \uncover<2-> {
    \node[fast,label=right:\pythontime] at (3,\pythontime) {};
    \node[thing] at (3,-2) {Python 3.4};

    \node[fast,label=right:\nodejstime] at (5,\nodejstime) {};
    \node[thing] at (5,-1) {node.js};
    }

    \uncover<3-> {
    \node[slow,label=right:\phptime] (php) at (7,\phptime) {};
    \node[thing] at (7,-2) {PHP5};

    \node[slow,label=right:\javaopenjdktime] (java) at (9,\javaopenjdktime) {};
    \node[thing] at (9,-1) {Java OpenJDK};

    \node[slow,label=right:\golangtime] (golang) at (11,\golangtime) {};
    \node[thing] at (11,-2) {Go};
    }

    \uncover<4-> {
    \node[fast,label=right:\phppatchtime] (phppatch) at (7,\phppatchtime) {};
    \draw [->] (php) -- (phppatch);
    % pull req https://github.com/php/php-src/pull/1387
    }

    \visible<5> {
    \node[maybefast,label=right:?] (javamaybefast) at (9,\javaopenjdktime*0.5) {};
    \draw [maybe,->] (java) -- (javamaybefast);

    \node[maybefast,label=right:?] (golangmaybefast) at (11,\golangtime*0.5) {};
    \draw [maybe,->] (golang) -- (golangmaybefast);
    }

    \uncover<6-> {
    \node[fast,label=right:\superfasttime] at (13,\superfasttime) {};
    \node[thing] at (13,-1) {fastpbkdf2};
    }

  \end{tikzpicture}
  
  \caption{PBKDF2-HMAC-SHA1, one block output, $2^{22}$ iterations}
  \end{figure}

}

\frame
{
  \frametitle{Parting thoughts...}

  \begin{itemize}
    \item<1-> PBKDF2 is a poor design, and described in an unhelpful way by its authors.
    \item<2-> Most implementations waste time and power.
    \item<3-> If you use PBKDF2, you can probably drop in a faster implementation and either
              increase security margin, or improve performance.
  \end{itemize}
}

\frame
{
  \frametitle{Thank you!}
  Questions?

  \vspace{5em}

  Twitter: @jpixton

  Mail: jbp@jbp.io

  Web: https://jbp.io/

  Slides: https://github.com/ctz/talks/

  Code: https://github.com/ctz/pbkdf2/
}

\end{document}
